\documentclass{beamer}
\usetheme{metropolis}

\usepackage{booktabs}
\usepackage{graphicx}
\usepackage{hyperref}

\title{When Data Talks, We Let AI Listen}
\subtitle{Using NLP to Predict the Severity of Software Vulnerabilities}
\author{Léa ULUSAN}
\institute{CIRCL - Computer Incident Response Center Luxembourg}
\date{BSides Luxembourg, June 19, 2025}

\begin{document}

\maketitle

% Slide 2: Agenda
\begin{frame}{Agenda}
    \begin{itemize}
        \item What is Vulnerability Lookup?
        \item Challenges with vulnerability data
        \item How we trained an AI model
        \item Real-world usage
        \item What's next?
    \end{itemize}
\end{frame}

% Slide 3: Project
\begin{frame}{Who is behind Vulnerability Lookup?}
    \begin{itemize}
        \item Open-source project by CIRCL
        \item Co-funded by the European Union
        \item Used by SOCs, CSIRTs, ENISA, and more
        \item Helps contextualize and enrich vulnerability data
    \end{itemize}
\end{frame}

% Slide 4: Motivation
\begin{frame}{Why go beyond CVE and NVD?}
    \begin{itemize}
        \item Legacy tool: \texttt{cve-search}
        \item NVD is no longer the only source
        \item Real-world vulnerabilities emerge from:
            \begin{itemize}
                \item GitHub advisories
                \item PyPI metadata
                \item CSAF vendor feeds
                \item Security blogs
            \end{itemize}
        \item We needed more scalability and correlation
    \end{itemize}
\end{frame}

% Slide 5: Enhancements
\begin{frame}{Making vulnerability data actionable}
    \begin{itemize}
        \item Collaborative features:
            \begin{itemize}
                \item Tags, bundles, and comments
                \item Sightings (real-world observations)
            \end{itemize}
        \item But: No CVSS scores available for many entries
        \item \textbf{Goal: Predict severity based only on description}
    \end{itemize}
\end{frame}

% Slide 6: Dataset
\begin{frame}{A constantly updated open-source dataset}
    \begin{itemize}
        \item >1 million correlated vulnerability records
        \item Extracted from CIRCL's internal data pipeline
        \item Fields:
            \begin{itemize}
                \item CVE ID, description, CVSS score/vector
                \item Tags, CPEs, CWEs, timestamps
            \end{itemize}
        \item Published on Hugging Face: \texttt{CIRCL/vulnerability-scores}
    \end{itemize}
\end{frame}

% Slide 7: Model
\begin{frame}{Severity prediction with RoBERTa}
    \begin{itemize}
        \item Fine-tuned RoBERTa model (multi-class classification)
        \item Input: CVE text description
        \item Output: Severity level (Low / Medium / High / Critical)
        \item Accuracy >90\% on recent CVEs
        \item Example use case:
            \begin{itemize}
                \item CVE-2024-12345 → predicted as Critical with 98\% confidence
            \end{itemize}
    \end{itemize}
\end{frame}

% Slide 8: Deployment
\begin{frame}{Production-ready: }
    \begin{itemize}
        \item FastAPI microservice to run models locally
        \item REST API: send description, get predicted severity
        \item Works offline — no sensitive data sent to the cloud
        \item Integrated into:
            \begin{itemize}
                \item \texttt{vulnlookup-cli}
                \item \texttt{vulnlookup-web}
                \item CIRCL’s internal enrichment pipeline
            \end{itemize}
    \end{itemize}
\end{frame}

% Slide 9: What's next
\begin{frame}{What’s next?}
    \begin{itemize}
        \item Improve CVE-to-CPE inference
        \item Predict vulnerability types (CWEs)
        \item Map to MITRE ATT\&CK tactics
        \item Better prioritization for patching
        \item Community contributions welcome!
    \end{itemize}
\end{frame}

% Slide 10: References
\begin{frame}{Open resources}
    \begin{itemize}
        \item Project website: \url{https://vulnerability.circl.lu}
        \item Dataset: \url{https://huggingface.co/datasets/CIRCL/vulnerability-scores}
        \item GitHub: \url{https://github.com/CIRCL}
        \item License: Free and open-source (various)
    \end{itemize}
\end{frame}

% Slide 11: Thank you
\begin{frame}{Thank you!}
    \begin{center}
        Questions?\\[1em]
        \texttt{lea.ulusan@circl.lu} \\
        \texttt{@VulnLookup} on Mastodon \\
        \texttt{https://vulnerability.circl.lu}
    \end{center}
\end{frame}

\end{document}
